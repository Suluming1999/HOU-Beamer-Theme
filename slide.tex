% this template mostly comes from Trinkle23897
% his github repo is https://github.com/Trinkle23897/THU-Beamer-Theme
% thanks a lot Trinkle!
% two color theme fdu_red and fdu_blue, choose your preference main document and switch in the menu
% github repo https://github.com/milanmarks/FDU-Beamer-Theme
\documentclass{beamer}
\usepackage{ctex, hyperref}
\usepackage[T1]{fontenc}
\usepackage[utf8]{vietnam}
\usepackage[vietnamese]{babel}

% other packages
\usepackage{latexsym,amsmath,xcolor,multicol,booktabs,calligra}
\usepackage{graphicx,pstricks,listings,stackengine}

% 设置字体
%\setCJKmainfont{Source Han Serif CN}
%\setmainfont{Times New Roman}
%\setsansfont{Fira Code}

%\setmonofont[Mapping={}]{CMU Typewriter Text}	%英文引号之类的正常显示,相当于设置英文字体
%\setsansfont{CMU Typewriter Text} %设置英文字体 Monaco, Consolas,  Fantasque Sans Mono
%\setmainfont{CMU Typewriter Text} %设置英文字体

%\setmonofont[Mapping={}]{Iosevka}
%\setsansfont{Iosevka}
%\setmainfont{Iosevka}

%\setmonofont[Mapping={}]{JetBrains Mono}
%\setsansfont{JetBrains Mono}
%\setmainfont{JetBrains Mono}

\setmonofont[Mapping={}]{Victor Mono}
\setsansfont{Victor Mono}
\setmainfont{Victor Mono}


\author{Nguyễn \& Tô}
\title{HOU Beamer Theme}

%\subtitle{毕业论文开题报告}
\subtitle{Báo cáo đề xuất luận án}

%\institute{中国人民大学国际关系学院}
\institute{Trường Đại học Mở Hà Nội}

%\date{2020年5月9日}
\date{Ngày 22 tháng 8 năm 2024}
\usepackage{RenminUniv}


% defs
\def\cmd#1{\texttt{\color{red}\footnotesize $\backslash$#1}}
\def\env#1{\texttt{\color{blue}\footnotesize #1}}
\definecolor{deepblue}{rgb}{0,0,0.5}
\definecolor{deepred}{rgb}{0.6,0,0}
\definecolor{deepgreen}{rgb}{0,0.5,0}
\definecolor{halfgray}{gray}{0.55}

\lstset{
    basicstyle=\ttfamily\small,
    keywordstyle=\bfseries\color{deepblue},
    emphstyle=\ttfamily\color{deepred},    % Custom highlighting style
    stringstyle=\color{deepgreen},
    numbers=left,
    numberstyle=\small\color{halfgray},
    rulesepcolor=\color{red!20!green!20!blue!20},
    frame=shadowbox,
}


\begin{document}

\kaishu
\begin{frame}
    \titlepage
    \begin{figure}[htpb]
        \begin{center}
            \includegraphics[width=0.4\linewidth]{pic/1.png}
        \end{center}
    \end{figure}
\end{frame}

\begin{frame}
    \tableofcontents[sectionstyle=show,subsectionstyle=show/shaded/hide,subsubsectionstyle=show/shaded/hide]
\end{frame}


\section{Bối cảnh chủ đề} % 课题背景

%\begin{otherlanguage*}{vietnamese}
\begin{frame}{Why Beamer?}
    \begin{itemize}[<+-| alert@+>] % 当然,除了alert,手动在里面插 \pause 也行
        % \item 大家都会\LaTeX{},好多学校都有自己的Beamer主题
        \item Tất cả sinh viên đại học nên nắm vững \LaTeX{},Nhiều trường có mẫu Beamer riêng.
        % \item 中文支持请选择 Xe\LaTeX{} 编译选项
        \item Để hỗ trợ ngôn ngữ tiếng Việt, vui lòng chọn tùy chọn biên dịch Xe\LaTeX{}
        % \item GitHub项目地址位于 \url{https://github.com/Suluming1999/HOU-Beamer-Theme},如果有bug或者feature request可以去里面提issue
        \item Địa chỉ dự án GitHub là tại \url{https://github.com/Suluming1999/HOU-Beamer-Theme},Nếu có lỗi, bạn có thể gửi ISSUS ở đó.
    \end{itemize}
\end{frame}
%\end{otherlanguage*}

\section{Tình trạng nghiên cứu} % 研究现状

%\subsection{Beamer主题分类}
\subsection{Beamer Phân loại chủ đề}
\begin{frame}
    \begin{itemize}
        % \item 有一些 \LaTeX{} 自带的
        \item \LaTeX{} đi kèm với một số mẫu chủ đề
        \item Mẫu này đến từ: \newline \url{https://www.latexstudio.net/archives/4051.html}
        %\item 但是最初的 \href{http://far.tooold.cn/post/latex/beamertsinghua}{\color{purple}{link}} \cite{origin} 已经失效了
        \item Mẫu nguyên bản nhất \href{http://far.tooold.cn/post/latex/beamertsinghua}{\color{purple}{link}} \cite{origin} Đã hết hạn.
        %\item 本模板在Trinkle23897的THU-Beamer-Theme基础上修改而成,感谢苏鹿鸣与 Nguyễn Mơ同学!\href{https://github.com/Trinkle23897/THU-Beamer-Theme}{\color{red}{戳我}}
        \item Mẫu này được Nguyễn \& Tô chỉnh sửa dựa trên THU-Beamer-Theme!\href{https://github.com/Suluming1999/HOU-Beamer-Theme}{\color{red}{bấm}}
        
    \end{itemize}
\end{frame}


\section{Nội dung nghiên cứu} % 研究内容

%\subsection{美化主题}
\subsection{Làm đẹp chủ đề}
%\begin{frame}{这一份主题与原始的THU Beamer Theme区别在于}
\begin{frame}{Sự khác biệt giữa chủ đề này và chủ đề THU Beamer gốc là ở chỗ}
    \begin{itemize}
        % \item 顶栏的小点变成一行而不是多行
        \item Các dấu chấm ở thanh trên cùng trở thành một dòng thay vì nhiều dòng
        \item Tiếng Việt sử dụng phông chữ CMU Typewriter.
        \item Để biết thêm các chức năng của mẫu này, vui lòng tham khảo: \url{https://www.latexstudio.net/archives/4051.html}
        \item Sau đây liệt kê một số ứng dụng của Beamer, một số đoạn trích từ: \url{https://tuna.moe/event/2018/latex/}
    \end{itemize}
\end{frame}

% \subsection{如何更好地做Beamer}
\subsection{Làm thế nào để trở thành một Beamer tốt hơn}
\begin{frame}{Why Beamer}
    \begin{itemize}
        % \item \LaTeX 广泛用于学术界,期刊会议论文模板
        \item \LaTeX\; được sử dụng rộng rãi trong giới học thuật, mẫu bài báo hội nghị tạp chí
    \end{itemize}
    \begin{table}[H]
        \centering
        % \scriptsize
        \tiny
        \begin{tabular}{c|c}
            Microsoft\textsuperscript{\textregistered}  Word & \LaTeX \\
            \hline
            % 文字处理工具 & 专业排版软件 \\
            Công cụ xử lý văn bản & Phần mềm sắp chữ chuyên nghiệp \\
            %容易上手,简单直观 & 容易上手 \\
            Dễ sử dụng, đơn giản và trực quan & Dễ dàng sử dụng \\
            % 所见即所得 & 所见即所想,所想即所得 \\
            Những gì bạn thấy là những gì bạn nhận được & Yêu cầu lập trình để xác định \\
            % 高级功能不易掌握 & 进阶难,但一般用不到 \\
            Các chức năng nâng cao rất khó để thành thạo & Khó thăng tiến nhưng nhìn chung là không cần thiết \\
            % 处理长文档需要丰富经验 & 和短文档处理基本无异 \\
            Xử lý văn bản dài đòi hỏi kinh nghiệm sâu rộng & Về cơ bản giống như xử lý tài liệu ngắn \\
            % 花费大量时间调格式 & 无需担心格式,专心作者内容 \\
            Dành nhiều thời gian để điều chỉnh định dạng & Chỉ cần tập trung vào việc tạo nội dung \\
            % 公式排版差强人意 & 尤其擅长公式排版 \\
            Việc sắp chữ công thức không đạt yêu cầu & Soạn công thức tốt \\
            % 二进制格式,兼容性差 & 文本文件,易读、稳定 \\
            Định dạng nhị phân, khả năng tương thích kém & Tệp văn bản, dễ đọc và ổn định \\
            % 付费商业许可 & 自由免费使用 \\
            Giấy phép thương mại trả phí & miễn phí sử dụng\\
        \end{tabular}
    \end{table}
\end{frame}

\begin{frame}{Ví dụ về sắp chữ}  % 排版举例
	\text{\textcolor{purple}{Công thức không đánh số}}
    %\begin{exampleblock}{Công thức không đánh số} % 无编号公式,记得 加 * 
      %  \begin{equation*}
        $$J(\theta)=\mathbb{E}_{\pi_\theta}[G_t]=\sum_{s\in\mathcal{S}}d^\pi(s)V^\pi(s)=\sum_{s\in\mathcal{S}}d^\pi(s)\sum_{a\in\mathcal{A}}\pi_\theta(a|s)Q^\pi(s,a)$$
       % \end{equation*}
    %\end{exampleblock}
    % \begin{exampleblock}{多行多列公式\footnote{如果公式中有文字出现,请用 $\backslash$mathrm\{\} 或者 $\backslash$text\{\} 包含,不然就会变成 $clip$,在公式里看起来比 $\mathrm{clip}$ 丑非常多。}}
    % \begin{exampleblock}{Công thức nhiều hàng và nhiều cột\footnote{Nếu có văn bản trong công thức, vui lòng sử dụng $\backslash$mathrm\{\} hoặc $\backslash$text\{\} để đưa nó vào, nếu không nó sẽ trở thành một đoạn $clip$, trông xấu hơn nhiều so với $\mathrm{clip}$ trong công thức.}}
   
    \begin{exampleblock}{Công thức nhiều hàng và nhiều cột\footnote{Nếu có văn bản trong công thức, vui lòng sử dụng $\backslash$mathrm\{\} hoặc $\backslash$text\{\} để đưa văn bản đó vào.}}
        % 使用 & 分隔
        \begin{align}
            Q_\mathrm{target}&=r+\gamma Q^\pi(s^\prime, \pi_\theta(s^\prime)+\epsilon)\\
            \epsilon&\sim\mathrm{clip}(\mathcal{N}(0, \sigma), -c, c)\nonumber
        \end{align}
    \end{exampleblock}
\end{frame}

\begin{frame}
    \begin{exampleblock}{Công thức nhiều dòng được đánh số}  % 编号多行公式
        % Taken from Mathmode.tex
        \begin{multline}
            A=\lim_{n\rightarrow\infty}\Delta x\left(a^{2}+\left(a^{2}+2a\Delta x+\left(\Delta x\right)^{2}\right)\right.\label{eq:reset}\\
            +\left(a^{2}+2\cdot2a\Delta x+2^{2}\left(\Delta x\right)^{2}\right)\\
            +\left(a^{2}+2\cdot3a\Delta x+3^{2}\left(\Delta x\right)^{2}\right)\\
            +\ldots\\
            \left.+\left(a^{2}+2\cdot(n-1)a\Delta x+(n-1)^{2}\left(\Delta x\right)^{2}\right)\right)\\
            =\frac{1}{3}\left(b^{3}-a^{3}\right)
        \end{multline}
    \end{exampleblock}
\end{frame}

\begin{frame}{Đồ họa và cột}
    % From thuthesis user guide.
    \begin{minipage}[c]{0.3\linewidth}
        \psset{unit=0.8cm}
        \begin{pspicture}(-1.75,-3)(3.25,4)
            \psline[linewidth=0.25pt](0,0)(0,4)
            \rput[tl]{0}(0.2,2){$\vec e_z$}
            \rput[tr]{0}(-0.9,1.4){$\vec e$}
            \rput[tl]{0}(2.8,-1.1){$\vec C_{ptm{ext}}$}
            \rput[br]{0}(-0.3,2.1){$\theta$}
            \rput{25}(0,0){%
            \psframe[fillstyle=solid,fillcolor=lightgray,linewidth=.8pt](-0.1,-3.2)(0.1,0)}
            \rput{25}(0,0){%
            \psellipse[fillstyle=solid,fillcolor=yellow,linewidth=3pt](0,0)(1.5,0.5)}
            \rput{25}(0,0){%
            \psframe[fillstyle=solid,fillcolor=lightgray,linewidth=.8pt](-0.1,0)(0.1,3.2)}
            \rput{25}(0,0){\psline[linecolor=red,linewidth=1.5pt]{->}(0,0)(0.,2)}
%           \psRotation{0}(0,3.5){$\dot\phi$}
%           \psRotation{25}(-1.2,2.6){$\dot\psi$}
            \psline[linecolor=red,linewidth=1.25pt]{->}(0,0)(0,2)
            \psline[linecolor=red,linewidth=1.25pt]{->}(0,0)(3,-1)
            \psline[linecolor=red,linewidth=1.25pt]{->}(0,0)(2.85,-0.95)
            \psarc{->}{2.1}{90}{112.5}
            \rput[bl](.1,.01){C}
        \end{pspicture}
    \end{minipage}\hspace{1cm}
    \begin{minipage}{0.5\linewidth}
        \medskip
        %\hspace{2cm}
        \begin{figure}[h]
            \centering
            \includegraphics[height=.4\textheight]{pic/dtmf.pdf}
        \end{figure}
    \end{minipage}
\end{frame}

\begin{frame}[fragile]{\LaTeX{} Các lệnh chung}  % 常用命令
    \begin{exampleblock}{Lệnh}  % 命令
        \centering
        \footnotesize
        \begin{tabular}{llll}
            \cmd{chapter} & \cmd{section} & \cmd{subsection} & \cmd{paragraph} \\
            chương & 节 & 小节 & 带题头段落 \\\hline
            \cmd{centering} & \cmd{emph} & \cmd{verb} & \cmd{url} \\
            居中对齐 & 强调 & 原样输出 & 超链接 \\\hline
            \cmd{footnote} & \cmd{item} & \cmd{caption} & \cmd{includegraphics} \\
            脚注 & 列表条目 & 标题 & 插入图片 \\\hline
            \cmd{label} & \cmd{cite} & \cmd{ref} \\
            标号 & 引用参考文献 & 引用图表公式等\\\hline
        \end{tabular}
    \end{exampleblock}
    \begin{exampleblock}{môi trường} % 环境
        \centering
        \footnotesize
        \begin{tabular}{lll}
            \env{table} & \env{figure} & \env{equation}\\
            表格 & 图片 & 公式 \\\hline
            \env{itemize} & \env{enumerate} & \env{description}\\
            无编号列表 & 编号列表 & 描述 \\\hline
        \end{tabular}
    \end{exampleblock}
\end{frame}

\begin{frame}[fragile]{\LaTeX{} Ví dụ về lệnh môi trường} % 环境命令举例
    \begin{minipage}{0.5\linewidth}
\begin{lstlisting}[language=TeX]
\begin{itemize}
  \item A \item B
  \item C
  \begin{itemize}
    \item C-1
  \end{itemize}
\end{itemize}
\end{lstlisting}
    \end{minipage}\hspace{1cm}
    \begin{minipage}{0.3\linewidth}
        \begin{itemize}
            \item A
            \item B
            \item C
            \begin{itemize}
                \item C-1
            \end{itemize}
        \end{itemize}
    \end{minipage}
    \medskip
    \pause
    \begin{minipage}{0.5\linewidth}
\begin{lstlisting}[language=TeX]
\begin{enumerate}
  \item quốc gia \item 666
  \item xã hội
  \begin{itemize}
    \item[n+e] trụ cột
  \end{itemize}
\end{enumerate}
\end{lstlisting}
    \end{minipage}\hspace{1cm}
    \begin{minipage}{0.3\linewidth}
        \begin{enumerate}
            \item quốc gia  % 国民
            \item 666  % 幸福
            \item xã hội  % 社会
            \begin{itemize}
                \item[n+e] trụ cột  % 栋梁
            \end{itemize}
        \end{enumerate}
    \end{minipage}
\end{frame}

\begin{frame}[fragile]{\LaTeX{} công thức toán học}
    \begin{columns}
        \begin{column}{.55\textwidth}
\begin{lstlisting}[language=TeX]
$V = \frac{4}{3}\pi r^3$

\[
  V = \frac{4}{3}\pi r^3
\]

\begin{equation}
  \label{eq:vsphere}
  V = \frac{4}{3}\pi r^3
\end{equation}
\end{lstlisting}
        \end{column}
        \begin{column}{.4\textwidth}
            $V = \frac{4}{3}\pi r^3$
            \[
                V = \frac{4}{3}\pi r^3
            \]
            \begin{equation}
                \label{eq:vsphere}
                V = \frac{4}{3}\pi r^3
            \end{equation}
        \end{column}
    \end{columns}
    \begin{itemize}
        \item Để biết thêm thông tin xin vui lòng xem \href{https://vi.wikipedia.org/wiki/Tr%E1%BB%A3_gi%C3%BAp:To%C3%A1n_h%E1%BB%8Dc}{\color{purple}{đây}}
    \end{itemize}
\end{frame}

\begin{frame}[fragile]
    \begin{columns}
        \column{.6\textwidth}
\begin{lstlisting}[language=TeX]
    \begin{table}[htbp]
      \caption{编号与含义}
      \label{tab:number}
      \centering
      \begin{tabular}{cl}
        \toprule
        编号 & 含义 \\
        \midrule
        1 & 4.0 \\
        2 & 3.7 \\
        \bottomrule
      \end{tabular}
    \end{table}
    公式~(\ref{eq:vsphere}) 的
    编号与含义请参见
    表~\ref{tab:number}。
\end{lstlisting}
        \column{.4\textwidth}
        \begin{table}[htpb]
            \centering
            \caption{编号与含义}
            \label{tab:number}
            \begin{tabular}{cl}\toprule
                编号 & 含义 \\\midrule
                1 & 4.0\\
                2 & 3.7\\\bottomrule
            \end{tabular}
        \end{table}
        \normalsize 公式~(\ref{eq:vsphere})的编号与含义请参见表~\ref{tab:number}。
    \end{columns}
\end{frame}

\begin{frame}{作图}
    \begin{itemize}
        \item 矢量图 eps, ps, pdf
        \begin{itemize}
            \item METAPOST, pstricks, pgf $\ldots$
            \item Xfig, Dia, Visio, Inkscape $\ldots$
            \item Matlab / Excel 等保存为 pdf
        \end{itemize}
        \item 标量图 png, jpg, tiff $\ldots$
        \begin{itemize}
            \item 提高清晰度,避免发虚
            \item 应尽量避免使用
        \end{itemize}
    \end{itemize}
    \begin{figure}[htpb]
        \centering
        \includegraphics[width=0.2\linewidth]{pic/Renmin_Univ_Logo.eps}
        \caption{这个校徽就是矢量图}
    \end{figure}
\end{frame}

\section{Tiến độ kế hoạch}  % 计划进度
\begin{frame}
    \begin{itemize}
        \item 一月:完成文献调研
        \item 二月:复现并评测各种Beamer主题美观程度
        \item 三、四月:美化TOU Beamer主题
        \item 五月:论文撰写
    \end{itemize}
\end{frame}

\section{Tài liệu tham khảo} % 参考文献
\begin{frame}[allowframebreaks]
    \bibliography{ref}
    \bibliographystyle{alpha}
    % 如果参考文献太多的话,可以像下面这样调整字体:
    % \tiny\bibliographystyle{alpha}
\end{frame}

\begin{frame}
    \begin{center}
        {\Huge\calligra Thanks!}
    \end{center}
\end{frame}

\end{document}